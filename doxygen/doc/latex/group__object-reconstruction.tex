\section{object-\/reconstruction}
\label{group__object-reconstruction}\index{object-\/reconstruction@{object-\/reconstruction}}


A module to reconstruct in 3D a set of pixel and visualize the reconstruction.  


A module to reconstruct in 3D a set of pixel and visualize the reconstruction. 

\hypertarget{group__handIKModule_intro_sec}{}\subsection{Description}\label{group__handIKModule_intro_sec}
This module, given a set of pixels, returns the point cloud along with the minimum enclosing bounding box. It usually employs the graph\+Based\+Segmentation module, which provides all the pixels belonging to a certain segment of the image, but it can be used with any other segmentation module that retrieves the list of pixels belonging to the object to be reconstructed.\hypertarget{group__handIKModule_rpc_port}{}\subsection{Commands\+:}\label{group__handIKModule_rpc_port}
The commands sent as bottles to the module port /$<$mod\+Name$>$/rpc are described in the following\+:

{\bfseries H\+E\+LP} ~\newline
format\+: \mbox{[}help\mbox{]} ~\newline
action\+: it returns the list of the things to do to use the module properly.

{\bfseries P\+I\+X\+EL} ~\newline
format\+: \mbox{[}u v\mbox{]} ~\newline
action\+: the pixel is given to the segmentation module, which responds with the pixels belonging to the chosen segment.

{\bfseries R\+E\+C\+O\+N\+S\+T\+R\+U\+CT} ~\newline
format\+: \mbox{[}3\+Drec\mbox{]} \char`\"{}param1\char`\"{}~\newline
action\+: the set of pixels are projected in 3D and provided on the output port -- typically /object-\/reconstruction/mesh\+:o. The optional parameter \char`\"{}param1\char`\"{} can be set to \char`\"{}visualize\char`\"{}; in this case the reconstructed cloud will also be displayed using Point Cloud Library library.

{\bfseries C\+H\+A\+N\+GE N\+A\+ME} ~\newline
format\+: \mbox{[}name (new\+Name)\mbox{]}~\newline
action\+: The reconstructed pointclouds will be saved as new\+Name\+X.\+ply, where new\+Name is thename given by the user and X is number increasing for each reconstruction saved on the same root name. Default name is 3\+Dobject.

{\bfseries G\+ET} ~\newline
format\+: \mbox{[}get point (u v)\mbox{]}~\newline
action\+: The pixel (u v) is projected in 3D and returned in the format (x y z).

{\bfseries S\+ET} ~\newline
format\+: \mbox{[}set write param\mbox{]}~\newline
action\+: If param is set to on, the reconstructed point cloud will be written in the home context path folder. This functionality can be turned off by setting param to off.\hypertarget{group__handIKModule_lib_sec}{}\subsection{Libraries}\label{group__handIKModule_lib_sec}

\begin{DoxyItemize}
\item Y\+A\+RP libraries.
\item objects3D library
\item \hyperlink{group__minimumBoundingBox}{minimum\+Bounding\+Box} library
\item stereo-\/vision library
\item Open\+CV library
\item Point Cloud Library.
\end{DoxyItemize}\hypertarget{group__handIKModule_portsc_sec}{}\subsection{Ports Created}\label{group__handIKModule_portsc_sec}

\begin{DoxyItemize}
\item {\itshape /} $<$mod\+Name$>$/rpc remote procedure call. It always replies something.
\item {\itshape /} $<$mod\+Name$>$/segmentation this is the port through which the pixel belonging to the object is sent to the segmentation module, and the set of pixels belonging to the object are returned.
\item {\itshape /} $<$mod\+Name$>$/mesh\+:o this is the port where the reconstructed object is returned.
\end{DoxyItemize}\hypertarget{group__handIKModule_parameters_sec}{}\subsection{Parameters}\label{group__handIKModule_parameters_sec}
The following are the options that are usually contained in the configuration file\+:

--name {\itshape name} 
\begin{DoxyItemize}
\item specify the module name, which is {\itshape object-\/reconstruction} by default.
\end{DoxyItemize}

--robot {\itshape robot} 
\begin{DoxyItemize}
\item specify the robot that has to be used. It is icub by default.
\end{DoxyItemize}

-- compute\+BB {\itshape compute\+BB} 
\begin{DoxyItemize}
\item tells if the algorithm has also to compute the minimum enclosing bounding box. It can be on or off.
\end{DoxyItemize}\hypertarget{group__handIKModule_tested_os_sec}{}\subsection{Tested OS}\label{group__handIKModule_tested_os_sec}
Windows, Linux

\begin{DoxyAuthor}{Author}
Ilaria Gori 
\end{DoxyAuthor}
